%% bare_conf.tex
%% V1.4b
%% 2015/08/26
%% by Michael Shell
%% See:
%% http://www.michaelshell.org/
%% for current contact information.
%%
%% This is a skeleton file demonstrating the use of IEEEtran.cls
%% (requires IEEEtran.cls version 1.8b or later) with an IEEE
%% conference paper.
%%
%% Support sites:
%% http://www.michaelshell.org/tex/ieeetran/
%% http://www.ctan.org/pkg/ieeetran
%% and
%% http://www.ieee.org/

%%*************************************************************************
%% Legal Notice:
%% This code is offered as-is without any warranty either expressed or
%% implied; without even the implied warranty of MERCHANTABILITY or
%% FITNESS FOR A PARTICULAR PURPOSE! 
%% User assumes all risk.
%% In no event shall the IEEE or any contributor to this code be liable for
%% any damages or losses, including, but not limited to, incidental,
%% consequential, or any other damages, resulting from the use or misuse
%% of any information contained here.
%%
%% All comments are the opinions of their respective authors and are not
%% necessarily endorsed by the IEEE.
%%
%% This work is distributed under the LaTeX Project Public License (LPPL)
%% ( http://www.latex-project.org/ ) version 1.3, and may be freely used,
%% distributed and modified. A copy of the LPPL, version 1.3, is included
%% in the base LaTeX documentation of all distributions of LaTeX released
%% 2003/12/01 or later.
%% Retain all contribution notices and credits.
%% ** Modified files should be clearly indicated as such, including  **
%% ** renaming them and changing author support contact information. **
%%*************************************************************************


% *** Authors should verify (and, if needed, correct) their LaTeX system  ***
% *** with the testflow diagnostic prior to trusting their LaTeX platform ***
% *** with production work. The IEEE's font choices and paper sizes can   ***
% *** trigger bugs that do not appear when using other class files.       ***                          ***
% The testflow support page is at:
% http://www.michaelshell.org/tex/testflow/



\documentclass[10pt, conference, letterpaper]{IEEEtran}


  	\usepackage[pdftex]{graphicx}
  	\graphicspath{{../pdf/}{../jpeg/}}
	\DeclareGraphicsExtensions{.pdf,.jpeg,.png}

	\usepackage[cmex10]{amsmath}
	\usepackage{mathabx}
	\usepackage{algorithmic}
	\usepackage{array}
	\usepackage{mdwmath}
	\usepackage{mdwtab}
	\usepackage{eqparbox}
	\usepackage{url}
	\usepackage{color}
	\hyphenation{op-tical net-works semi-conduc-tor}
	\usepackage{xcolor}
	\usepackage[export]{adjustbox}
	\usepackage{multirow}
	\usepackage{listings}

\usepackage{color}

\lstdefinelanguage{p4}{
  sensitive = true,
  keywords={if, else, default, key, actions, size, default_action},
  keywords=[2]{action, apply, bit, const, control, default, enum, error, extern,         false, header,header_union, in, inout, int, package, parser, out, select, state,     struct, table,transition, true, typedef, varbit, verify, metadata, header_type,      fields},
  keywords=[3]{exact,lpm,range,selection,ternary},
  keywords=[4]{.apply()},
  keywordstyle=\color{blue},
  keywordstyle=[2]\color{purple}\bfseries,
  keywordstyle=[3]\color{green!50!black}\bfseries,
  keywordstyle=[4]\color{black},
  identifierstyle=\color{black},
  sensitive=false,
  comment=[l]{//},
  morecomment=[s]{/*}{*/},
  commentstyle=\color{green!40!black}\ttfamily,
  stringstyle=\color{red}\ttfamily,
  morestring=[b]',
  morestring=[b]",
}

\lstset{
  language=p4,
  extendedchars=true,
  basicstyle=\footnotesize\ttfamily,
  showstringspaces=false,
  showspaces=false,
  numbers=none,
  numberstyle=\footnotesize,
  numbersep=9pt,
  tabsize=2,
  breaklines=true,
  showtabs=false,
  captionpos=b
}


% *** MISC UTILITY PACKAGES ***
%
%\usepackage{ifpdf}
% Heiko Oberdiek's ifpdf.sty is very useful if you need conditional
% compilation based on whether the output is pdf or dvi.
% usage:
% \ifpdf
%   % pdf code
% \else
%   % dvi code
% \fi
% The latest version of ifpdf.sty can be obtained from:
% http://www.ctan.org/pkg/ifpdf
% Also, note that IEEEtran.cls V1.7 and later provides a builtin
% \ifCLASSINFOpdf conditional that works the same way.
% When switching from latex to pdflatex and vice-versa, the compiler may
% have to be run twice to clear warning/error messages.






% *** CITATION PACKAGES ***
%
%\usepackage{cite}
% cite.sty was written by Donald Arseneau
% V1.6 and later of IEEEtran pre-defines the format of the cite.sty package
% \cite{} output to follow that of the IEEE. Loading the cite package will
% result in citation numbers being automatically sorted and properly
% "compressed/ranged". e.g., [1], [9], [2], [7], [5], [6] without using
% cite.sty will become [1], [2], [5]--[7], [9] using cite.sty. cite.sty's
% \cite will automatically add leading space, if needed. Use cite.sty's
% noadjust option (cite.sty V3.8 and later) if you want to turn this off
% such as if a citation ever needs to be enclosed in parenthesis.
% cite.sty is already installed on most LaTeX systems. Be sure and use
% version 5.0 (2009-03-20) and later if using hyperref.sty.
% The latest version can be obtained at:
% http://www.ctan.org/pkg/cite
% The documentation is contained in the cite.sty file itself.






% *** GRAPHICS RELATED PACKAGES ***
%
\ifCLASSINFOpdf
  % \usepackage[pdftex]{graphicx}
  % declare the path(s) where your graphic files are
  % \graphicspath{{../pdf/}{../jpeg/}}
  % and their extensions so you won't have to specify these with
  % every instance of \includegraphics
  % \DeclareGraphicsExtensions{.pdf,.jpeg,.png}
\else
  % or other class option (dvipsone, dvipdf, if not using dvips). graphicx
  % will default to the driver specified in the system graphics.cfg if no
  % driver is specified.
  % \usepackage[dvips]{graphicx}
  % declare the path(s) where your graphic files are
  % \graphicspath{{../eps/}}
  % and their extensions so you won't have to specify these with
  % every instance of \includegraphics
  % \DeclareGraphicsExtensions{.eps}
\fi
% graphicx was written by David Carlisle and Sebastian Rahtz. It is
% required if you want graphics, photos, etc. graphicx.sty is already
% installed on most LaTeX systems. The latest version and documentation
% can be obtained at: 
% http://www.ctan.org/pkg/graphicx
% Another good source of documentation is "Using Imported Graphics in
% LaTeX2e" by Keith Reckdahl which can be found at:
% http://www.ctan.org/pkg/epslatex
%
% latex, and pdflatex in dvi mode, support graphics in encapsulated
% postscript (.eps) format. pdflatex in pdf mode supports graphics
% in .pdf, .jpeg, .png and .mps (metapost) formats. Users should ensure
% that all non-photo figures use a vector format (.eps, .pdf, .mps) and
% not a bitmapped formats (.jpeg, .png). The IEEE frowns on bitmapped formats
% which can result in "jaggedy"/blurry rendering of lines and letters as
% well as large increases in file sizes.
%
% You can find documentation about the pdfTeX application at:
% http://www.tug.org/applications/pdftex





% *** MATH PACKAGES ***
%
%\usepackage{amsmath}
% A popular package from the American Mathematical Society that provides
% many useful and powerful commands for dealing with mathematics.
%
% Note that the amsmath package sets \interdisplaylinepenalty to 10000
% thus preventing page breaks from occurring within multiline equations. Use:
%\interdisplaylinepenalty=2500
% after loading amsmath to restore such page breaks as IEEEtran.cls normally
% does. amsmath.sty is already installed on most LaTeX systems. The latest
% version and documentation can be obtained at:
% http://www.ctan.org/pkg/amsmath





% *** SPECIALIZED LIST PACKAGES ***
%
%\usepackage{algorithmic}
% algorithmic.sty was written by Peter Williams and Rogerio Brito.
% This package provides an algorithmic environment fo describing algorithms.
% You can use the algorithmic environment in-text or within a figure
% environment to provide for a floating algorithm. Do NOT use the algorithm
% floating environment provided by algorithm.sty (by the same authors) or
% algorithm2e.sty (by Christophe Fiorio) as the IEEE does not use dedicated
% algorithm float types and packages that provide these will not provide
% correct IEEE style captions. The latest version and documentation of
% algorithmic.sty can be obtained at:
% http://www.ctan.org/pkg/algorithms
% Also of interest may be the (relatively newer and more customizable)
% algorithmicx.sty package by Szasz Janos:
% http://www.ctan.org/pkg/algorithmicx




% *** ALIGNMENT PACKAGES ***
%
%\usepackage{array}
% Frank Mittelbach's and David Carlisle's array.sty patches and improves
% the standard LaTeX2e array and tabular environments to provide better
% appearance and additional user controls. As the default LaTeX2e table
% generation code is lacking to the point of almost being broken with
% respect to the quality of the end results, all users are strongly
% advised to use an enhanced (at the very least that provided by array.sty)
% set of table tools. array.sty is already installed on most systems. The
% latest version and documentation can be obtained at:
% http://www.ctan.org/pkg/array


% IEEEtran contains the IEEEeqnarray family of commands that can be used to
% generate multiline equations as well as matrices, tables, etc., of high
% quality.




% *** SUBFIGURE PACKAGES ***
%\ifCLASSOPTIONcompsoc
%  \usepackage[caption=false,font=normalsize,labelfont=sf,textfont=sf]{subfig}
%\else
%  \usepackage[caption=false,font=footnotesize]{subfig}
%\fi
% subfig.sty, written by Steven Douglas Cochran, is the modern replacement
% for subfigure.sty, the latter of which is no longer maintained and is
% incompatible with some LaTeX packages including fixltx2e. However,
% subfig.sty requires and automatically loads Axel Sommerfeldt's caption.sty
% which will override IEEEtran.cls' handling of captions and this will result
% in non-IEEE style figure/table captions. To prevent this problem, be sure
% and invoke subfig.sty's "caption=false" package option (available since
% subfig.sty version 1.3, 2005/06/28) as this is will preserve IEEEtran.cls
% handling of captions.
% Note that the Computer Society format requires a larger sans serif font
% than the serif footnote size font used in traditional IEEE formatting
% and thus the need to invoke different subfig.sty package options depending
% on whether compsoc mode has been enabled.
%
% The latest version and documentation of subfig.sty can be obtained at:
% http://www.ctan.org/pkg/subfig




% *** FLOAT PACKAGES ***
%
%\usepackage{fixltx2e}
% fixltx2e, the successor to the earlier fix2col.sty, was written by
% Frank Mittelbach and David Carlisle. This package corrects a few problems
% in the LaTeX2e kernel, the most notable of which is that in current
% LaTeX2e releases, the ordering of single and double column floats is not
% guaranteed to be preserved. Thus, an unpatched LaTeX2e can allow a
% single column figure to be placed prior to an earlier double column
% figure.
% Be aware that LaTeX2e kernels dated 2015 and later have fixltx2e.sty's
% corrections already built into the system in which case a warning will
% be issued if an attempt is made to load fixltx2e.sty as it is no longer
% needed.
% The latest version and documentation can be found at:
% http://www.ctan.org/pkg/fixltx2e


%\usepackage{stfloats}
% stfloats.sty was written by Sigitas Tolusis. This package gives LaTeX2e
% the ability to do double column floats at the bottom of the page as well
% as the top. (e.g., "\begin{figure*}[!b]" is not normally possible in
% LaTeX2e). It also provides a command:
%\fnbelowfloat
% to enable the placement of footnotes below bottom floats (the standard
% LaTeX2e kernel puts them above bottom floats). This is an invasive package
% which rewrites many portions of the LaTeX2e float routines. It may not work
% with other packages that modify the LaTeX2e float routines. The latest
% version and documentation can be obtained at:
% http://www.ctan.org/pkg/stfloats
% Do not use the stfloats baselinefloat ability as the IEEE does not allow
% \baselineskip to stretch. Authors submitting work to the IEEE should note
% that the IEEE rarely uses double column equations and that authors should try
% to avoid such use. Do not be tempted to use the cuted.sty or midfloat.sty
% packages (also by Sigitas Tolusis) as the IEEE does not format its papers in
% such ways.
% Do not attempt to use stfloats with fixltx2e as they are incompatible.
% Instead, use Morten Hogholm'a dblfloatfix which combines the features
% of both fixltx2e and stfloats:
%
% \usepackage{dblfloatfix}
% The latest version can be found at:
% http://www.ctan.org/pkg/dblfloatfix




% *** PDF, URL AND HYPERLINK PACKAGES ***
%
%\usepackage{url}
% url.sty was written by Donald Arseneau. It provides better support for
% handling and breaking URLs. url.sty is already installed on most LaTeX
% systems. The latest version and documentation can be obtained at:
% http://www.ctan.org/pkg/url
% Basically, \url{my_url_here}.




% *** Do not adjust lengths that control margins, column widths, etc. ***
% *** Do not use packages that alter fonts (such as pslatex).         ***
% There should be no need to do such things with IEEEtran.cls V1.6 and later.
% (Unless specifically asked to do so by the journal or conference you plan
% to submit to, of course. )


% correct bad hyphenation here
\hyphenation{op-tical net-works semi-conduc-tor}

\newcommand{\giovanni}[1]{{\color{red}\tiny{\textbf{#1}}}}
\newcommand{\bruno}[1]{{\color{blue}\tiny{\textbf{#1}}}}
\newcommand{\gtodo}[2]{{\color{purple}#1: }{\color{red}\tiny{\textbf{#2}}}}

\begin{document}
%
% paper title
% Titles are generally capitalized except for words such as a, an, and, as,
% at, but, by, for, in, nor, of, on, or, the, to and up, which are usually
% not capitalized unless they are the first or last word of the title.
% Linebreaks \\ can be used within to get better formatting as desired.
% Do not put math or special symbols in the title.
%\title{In-Network flow/session machine learning classification}
%\title{A Methodology for In-Network Machine Learning}
%\title{Switches Can Do It! \\ Pushing Machine Learning to the Data Plane}

%\title{Switches Can Do It! \\ Enabling In-Network Classification to the Date Plane}

%\title{Proclaim: Programmable Data plane for In-Network Classification }

\title{Programmable Switches \\ for in-Networking Classification \vspace{0.5cm}}


% author names and affiliations
% use a multiple column layout for up to three different
% affiliations
\author{}
\if 0
\author{\IEEEauthorblockN{Author 1}
\IEEEauthorblockA{Federal University of Espírito Santo\\
Espírito Santo, Brazil\\
Email: mail@mail.com}
\and
\IEEEauthorblockN{Author 2}
\IEEEauthorblockA{Federal University of Espírito Santo\\
Espírito Santo, Brazil\\
Email: mail@mail.com}
\and
\IEEEauthorblockN{Author 3}
\IEEEauthorblockA{Federal University of Espírito Santo\\
Espírito Santo, Brazil\\
Email: mail@mail.com}}
\fi
% conference papers do not typically use \thanks and this command
% is locked out in conference mode. If really needed, such as for
% the acknowledgment of grants, issue a \IEEEoverridecommandlockouts
% after \documentclass

% for over three affiliations, or if they all won't fit within the width
% of the page, use this alternative format:
% 
%\author{\IEEEauthorblockN{Michael Shell\IEEEauthorrefmark{1},
%Homer Simpson\IEEEauthorrefmark{2},
%James Kirk\IEEEauthorrefmark{3}, 
%Montgomery Scott\IEEEauthorrefmark{3} and
%Eldon Tyrell\IEEEauthorrefmark{4}}
%\IEEEauthorblockA{\IEEEauthorrefmark{1}School of Electrical and Computer Engineering\\
%Georgia Institute of Technology,
%Atlanta, Georgia 30332--0250\\ Email: see http://www.michaelshell.org/contact.html}
%\IEEEauthorblockA{\IEEEauthorrefmark{2}Twentieth Century Fox, Springfield, USA\\
%Email: homer@thesimpsons.com}
%\IEEEauthorblockA{\IEEEauthorrefmark{3}Starfleet Academy, San Francisco, California 96678-2391\\
%Telephone: (800) 555--1212, Fax: (888) 555--1212}
%\IEEEauthorblockA{\IEEEauthorrefmark{4}Tyrell Inc., 123 Replicant Street, Los Angeles, California 90210--4321}}




% use for special paper notices
%\IEEEspecialpapernotice{(Invited Paper)}




% make the title area
\maketitle
% As a general rule, do not put math, special symbols or citations
% in the abstract
\begin{abstract}
Deploying accurate machine learning algorithms into a high-throughput networking environment is a challenging task. On one hand, machine learning has proved itself useful for traffic classification in many contexts (e.g., intrusion detection, application classification, and early heavy hitter identification).
On the other hand, most of the work in the area is related to post-processing (i.e., training and testing are performed offline on previously collected samples) or to scenarios where the traffic has to leave the data plane to be classified (i.e., high latency).
In this work, we tackle the problem of creating simple and reasonably accurate machine learning models that can be deployed into the data plane in a way that performance degradation is acceptable.
To that purpose, we introduce a framework and discuss issues related to the translation of simple models, for handling individual packets or flows, into the P4 language.
We validate our framework with an intrusion detection use case and by deploying a single decision tree into a Netronome SmartNIC (Agilio CX 2x10GbE). 
Our results show that high-accuracy is achievable (above 95\%) with minor performance degradation, even for a large number of flows.
\end{abstract}

% no keywords




% For peer review papers, you can put extra information on the cover
% page as needed:
% \ifCLASSOPTIONpeerreview
% \begin{center} \bfseries EDICS Category: 3-BBND \end{center}
% \fi
%
% For peerreview papers, this IEEEtran command inserts a page break and
% creates the second title. It will be ignored for other modes.
\IEEEpeerreviewmaketitle


\section{Introduction}

%Our Goal: propose a framework based on machine learning to classify network traffic using P4. We are not interested in packet classification, but flow/session.

% Magnos 

%Machine Learning (ML) has driven a technological revolution with an unprecedented number of applications that enable automation in wide range of domains. This has happened due to the explosion in the availability of data, significant improvements in ML techniques, and advancement in computing capabilities. %ML goal is essentially  to identify and exploit hidden patterns in “training” so that a program is created (i.e., model) that fits the data.

In the recent years, we have seen an increasingly interest in research applying Machine Learning (ML) techniques to networking problems \cite{boutaba2018comprehensive}. On one hand, it has been motivated by technological advances in networking, such as network programmability via Software-Defined Networking (SDN) \cite{P4}. On the other hand, recent advances in ML have made these techniques flexible and resilient to make them applicable to various real-world scenarios. While programmable switches have been proven to be useful for in-network computing \cite{In-net-computation}, machine learning within programmable switches have had little success so far \cite{xiong2019switches}. 

With the rise of in-networking computing, the interest in running ML within network devices is rapidly growing for multiple reasons. Firstly, switches offer very high performance. The latency through a switch is in the order of nanoseconds per packet \cite{Tofino}, while high-end ML accelerators operate at the scale of tens of microseconds to milliseconds per inference \cite{Tensor}. Also, there are programmable devices such as smart NICs (e.g. Netronome \cite{NETRONOME}, Cavium \cite{CAVIUM} and Mellanox \cite{MELLANOX}) that can be placed at servers for accelerating ML use-cases. 

Another important motivation to push ML to programmable devices is that the performance of distributed ML is bounded by time required to get data to and from nodes. Therefore, if a switch can classify at the same rate that it carries packets to nodes in a distributed system, then it will equal or outperform any single node \cite{xiong2019switches}. In practice, if a smartNIC is deployed at the edge, the devices allow to terminate data early, reducing the load on the network and improving user experience thanks to reduced latency \cite{PIaFFE}.  

One of the challenges to implement ML algorithms within network devices is the hardware implementation complexity required to support mathematical operations. Operations such as addition, xor or bit shifting are feasible, but multiplication, polynomials or logarithms are not pipelined well. However, the RMT architecture \cite{bosshart2013forwarding} has a flexible parser and a customizable match-action
engine. To process packets at high speed, this architecture has a multi-stage pipeline where packets flow at line rate. It
allows lookups in memory (SRAM and TCAM), manipulating packet metadata and stateful registers, and performing boolean and arithmetic operations using ALUs. We believe that with this new generation of programmable hardware, it is worth rising a  question: \textit{can we claim that programmable switches do in-network classification?}

In this paper, our focus is on deploying ML classification trained models showing how to express them into the existing P4 language primitives. More specifically, our contributions are:
\begin{itemize}
    \item we introduce an innovative framework that enables the transformation of decision-tree models into P4 language pipeline; and
    \item as proof-of-concept, we build an in-network classifier for an IDS (Intrusion Detection System). We validate our implementation by using the BMv2 emulator \cite{p4-bmv2} and by deploying it into a Netronome SmartNIC. 
\end{itemize}

\if 0
In this paper, our focus is on deploying ML classification trained models showing how to express them into the existing P4 language primitives. More specifically, we introduce an innovative framework that enables the transformation of decision-tree models into P4 language pipeline. As a second contribution, we build as proof-of-concept an in-network classification for an IDS use case, deploying it in a Netronome SmartNIC. 

Our evaluation results have shown that high-accuracy for traffic classification is achievable (above  95\%) with minor performance degradation. We demonstrate a clear trade-off between accuracy and efficiency when choosing a per-packet or per-flow model. Classifying a flow  leads to more accurate results, at the cost of keeping tables' state (e.g. per-flow). In contrast, the per-packet model is stateless having lower accuracy but suffers from the issue of flow fragmentation.
\fi

We conducted extensive experiments in order to assess the models' quality and efficiency.
Our evaluation results have shown that high-accuracy for traffic classification is achievable (above  95\%) with minor performance degradation. 
We demonstrate a clear trade-off between accuracy and efficiency when choosing a per-packet or per-flow model. 
Classifying a flow  leads to more accurate results, at the cost of keeping tables' state, and it can be performed by observing only a few of its packets, i.e., when the flow is still ``young''. In contrast, the per-packet model is stateless having lower accuracy but suffering from the issue of flow fragmentation (i.e., different labels for packets in the same flow).


%Finally, we evaluate the classification feasibility, accuracy and performance.  

% Tem que melhorar o texto neste parágrafo.  Furthermore, we show that accuracy can be high even when classifying with a very few number of packets.

 %Furthermore, we show that as long as the set of features is static, updates to classification models can be deployed through the control plane alone, without changes to the data plane.

The remaining of this work is organized as follows. Section \ref{sec:methodology} presents the framework's architecture and implementation decisions. In Section \ref{sec:case}, we show how our methodology can be used to support intrusion detection. We position our work in the literature in Section \ref{sec:rw} and in Section \ref{sec:conc}, we present concluding remarks and directions for future work.

%The networking community has begun to lay the foundation for network programmability through fully programmable protocol-independent data planes (e.g.,the Barefoot Tofino chipset [5] and Netronome NICs [30]) with domain specific language (e.g., P4 [6]). These advances have allowed the data planes to support a programmatic network control, not only over forwarding (as SDN has enabled) but also over the collection of measurement data. 




\section{Programmable Data Plane \\ as a Classification Machine }\label{sec:methodology}

The goal of this section is to present the necessary steps to deploy a machine learning model into a programmable networking device by using the P4 language \cite{P4}.
To do so, the ``usual'' machine learning process has to be rethought once we want a fast and accurate classifier, under the limitations of the P4 language expressiveness and of the target hardware resources (e.g., memory and processing power).
Therefore, the success of our proposal depends on approaching the networking traffic classification problem at the right granularity (e.g., packets or flows) and choosing P4-suitable features and models.

Our framework can be seen as a shift from the standard match-action paradigm. Traditionally, switches' actions are driven by table matching, which can range from using only MAC addresses to more complex OpenFlow \cite{OpenFlow} rules. Now, P4 enables us to express the matching as the result of mathematical computations over features that can also be extracted from packet header and computed using e.g. stateful registers. We claim that such computations can be obtained from simple learning algorithms so that they can be helpful in different networking scenarios. One advantage of expressing the matching as a result of a learning algorithm is the possibility of having networking data-driven updates. In other words, if the nature of the traffic changes, new models can be built and deployed into the data plane.

\begin{figure}[t] %!t
\centering
\includegraphics[width=\linewidth]{figures/architecture.pdf}
\caption{Framework's Architecture}
\label{architecture}
\end{figure}

Figure \ref{architecture} shows the high-level architecture of our framework and its main components. The architecture is composed of three planes: 
\begin{enumerate}
    \item The \emph{Knowledge Plane} performs the transformation of knowledge into the a machine learning model. In this context, knowledge may refer to information obtained from external datasets, In-band Network Telemetry, or active measurements. The \emph{Feature Extractor} (Section \ref{sub:features}) component is in charge of selecting and transforming the knowledge base's fields into useful features, considering the limitations of the P4 language.
    The \emph{Machine Learning Model Builder} (Section \ref{sub:builder}) component trains the model based on the features that came from the previous component. 
    
    \item The \emph{Control Plane} connects the Knowledge and Data planes, and it has the goal of mapping the previously trained model to an application, given a target hardware architecture. When the \emph{ML to P4 Compiler} (Section \ref{sub:compiler}) component receives a model from the \emph{Machine Learning Model Builder} component, it creates a new application code based on the P4 language, and/or a new set of match-action rules. Then, the \emph{Firmware builder} component uses the application code to compile the firmware for a specific target architecture and sends it to the \emph{Agent Deployer} component, which, in turn, installs the new firmware into the physical device. Meanwhile, match-action rules, can be sent by the \emph{Controller} to the data plane via a southbound interface.

    \item In the \emph{Data Plane}, when the switch receives the new firmware and the set of match-action rules from the \emph{Control Plane}, it assumes a new behavior previously determined by the machine learning model. At this point, the switch recognizes the class that the packet or flow belongs to and reacts immediately to execute either forwarding or dropping policies.
    
    \if 0
    \item In the \emph{Data Plane}, despite the brand-new firmware generated by the \emph{Firmware Builder}, there is still the need in providing an abstraction layer that does the stitching between the bare-metal switch and control plane. Hence, the \emph{Agent Deployer} component is responsible for installing the new firmware into the device, which allows us to program in a fine-grained way the data plane. Once the firmware is installed, the device's matching behaviour is determined according to the machine learning model.
    \fi
\end{enumerate}

The following sections present details and considerations about the three most important components of our framework:  \emph{Feature Extractor}, \emph{Machine Learning Model Builder}, and \emph{ML to P4 Compiler}.

%\bruno{Smart Match-Action}
%\newline 
%Explain the architecture components
            


\subsection{Feature Extractor}\label{sub:features}

A crucial step in machine learning is to precisely identify the objects of study and the characteristics, i.e., features, that define those objects.
Traditionally, networking traffic data focus on two objects (or granularities), \emph{flows} or \emph{packets}, and features are extracted from headers and payloads.
The right choice of granularity and features heavily depends on the application. 

There is a trade-off when choosing between a per-packet or a per-flow model, especially in our context, where the goal is to build high-accuracy models capable of processing packets at line-rate. On one hand, per-flow features can be computed by aggregating information from several packets, and by exploring these richer features, it is possible to train better models than when using only packets. On the other hand, in order to classify a flow, it is necessary to keep up-to-date flows' states in memory, which can be resource intensive. Hence, a per-packet model has the potential to be more efficient, but it does not offer to the model aggregated measurements and the possibility of learning from temporal correlation.

\begin{figure*}[t] 
    \centering
        \includegraphics[scale=0.6]{figures/simple_tree_flow.pdf}
        \includegraphics[scale=0.6]{figures/simple_tree_pkt.pdf}
        \caption{Examples of decision trees for classifying flows (left) and packets (right).}
        \label{fig:dtexample}
\end{figure*}

Once the proper granularity is defined, in general, as many features as possible are extracted from the data. Then, feature selection and/or dimensionality reduction techniques are applied in order to restrict the set of candidates to a new set, with the more adequate and/or informative ones. 
Unfortunately, the processes of extracting and applying transformations to features commonly rely on operations (or primitives) that can not be easily (or efficiently) expressed in P4, our target language.

A canonical example of this issue is the lack of floating-point operations in P4. This characteristic of the language (and in many cases of the underlying hardware as well) disallows us to compute standard metrics (e.g., average and standard deviation) and to apply important data transformations (e.g., logarithm). Therefore, one must be aware of these limitations when building a model. Otherwise, it will not be possible to translate the model in question to a P4 application, and the task of in-network classification will not be achieved.
In Section \ref{sec:case}, we discuss how we overcame this issue when instantiating our framework for the scenario of intrusion detection.


\subsection{Machine Learning Model Builder}\label{sub:builder}

Once the dataset is ready, the next step is to build the classification model. The machine learning literature has a plethora of supervised learning techniques, but not all of them are suitable for our task in hand. More specifically, in addition to the usual requirements of a ML model (e.g., accuracy and generalization), here we are interested in building a model, $f$, that satisfies the following properties when classifying an instance $\mathbf{x}$:
\begin{enumerate}
    \item the necessary operations to compute $f(\mathbf{x})$ must be readily available in P4; and
    \item computing $f(\mathbf{x})$ must be fast, given a target hardware.
\end{enumerate}

For our framework, we decided to use a decision tree in an attempt to satisfy these two properties. It is out of the scope of this work to go into the details of many algorithms to justify our choice. Even so, we provide the main reasons for not choosing the following popular options: $k$-Nearest Neighbors ($k$-NN), Naive Bayes (NB), Support Vector Machine (SVM), and Artificial Neural Network (ANN).
The interested reader can find more about these methods in the machine learning literature (e.g., \cite{mitchell97,bishop2006,hastie2001,Goodfellow-et-al-2016}).

The $k$-Nearest Neighbors is an instance-based learning algorithm, which means that the model is the training data itself. In addition, sophisticated data-structures are necessary to find the closest elements to a given $\mathbf{x}$, that we wish to classify. Hence, 
if the target hardware has limited memory and processing power, 
it is unlikely that the classification operation will satisfy Property (2).

In order to classify an instance $\mathbf{x}$, the Naive Bayes classifier yields the class that maximizes the conditional probability of observing a certain class given $\mathbf{x}$. It is known to be a fast classifier, which means that it satisfies Property (2). However, P4 lacks the exponentiation operation, which is necessary in some variations of algorithm, thus, violating Property (1). Even if using a NB variation that can be expressed in P4, we opted for not doing so because the algorithm is known to give poor classification results when the features in $\mathbf{x}$ are correlated.


There are two main versions of the SVM classifier: with a linear or with a non-linear kernel (e.g., Radial Basis Function -- RBF). The former satisfies Properties (1) and (2), but it is only appropriate for linearly-separable problems. The latter is able to deal with more complex problems, but it may violate Property (1), if the kernel demands certain operations (e.g., exponentiation), and Property (2), due to the need of storing all support vectors and executing many kernel computations for classifying a single $\mathbf{x}$.

Similarly to a SVM with a non-linear kernel, classifying a given $\mathbf{x}$ with an ANN is challenging in our scenario. The classification can be costly, for involving matrix-vector multiplication operations, and it may depend on activation functions that are not available in P4.


We do not claim that it is not feasible to optimize/adapt/approximate the methods aforementioned to perform in-network classification, but we do argue that a decision tree classifier is a more natural choice for such a task given the P4 language current primitives.
In order to classify an element $\mathbf{x}$ with a decision tree model, only comparisons are necessary, and those can be easily expressed in P4 through \emph{if-else} statements (more details in Section \ref{sub:compiler}).


\begin{lstlisting}[float=t,caption={Main portions of the P4 code obtained by applying the \emph{ML to P4 Compiler} component to the per-flow tree of Figure \ref{fig:dtexample}.},label={lst:p4flow},frame=tb]

...

table classtable {

    key = {
        meta.class: exact;
    }
    actions = {
        forward_by_class;
        ...
    }
    size = 512;
    default_action = NoAction;
}

...

apply {
    // extract and aggregate features from the packet's headers. The extracted features are hdr.ipv4.totalLen and hdr.tcp.dstPort
    extract_features(); 

    // compute a per-flow identifier by hashing (source IP, source Port, destination IP, destination Port, IPv4 Protocol)
    hash();  
    
    // update the per-flow features according to the packet's features and flow identifier. It also retrieves the flow's updated features
    update_features(); 
   
    // Apply the decision tree model to the flow
    if (meta.NumPackets <= 16.5)
        meta.class = 0;
    else    
        if (meta.MinPacketSize <= 42.0)
            meta.class = 1;
        else    
            if (meta.TcpDstPort <= 109.5)
                meta.class = 1;
            else
                meta.class = 0;
    
    classtable.apply();                
}

...

\end{lstlisting}


\begin{lstlisting}[float=t,caption={{Main portions of the P4 code obtained by applying the \emph{ML to P4 Compiler} component to the per-packet tree of Figure \ref{fig:dtexample}}.},label={lst:p4pkt},frame=tb]

...

table classtable {

    key = {
        meta.class: exact;
    }
    actions = {
        forward_by_class;
        ...
    }
    size = 512;
    default_action = NoAction;
}

...

apply {

    // Apply the decision tree model to the the packet. In this case, features are extracted directly from the packet's headers
    if (hdr.ipv4.totalLen <= 658.5)
        if (hdr.tcp.dstPort <= 137.0)
            if (hdr.ipv4.protocol <= 11.5)
                meta.class = 1;
            else
                meta.class = 0;
        else
            meta.class = 0;
    else
        meta.class = 0;
    
    classtable.apply();                
}

...

\end{lstlisting}

\if 0

\begin{lstlisting}[float=t,caption={Main portions of the P4 code obtained by applying the \emph{ML to P4 Compiler} component to the per-flow tree of Figure \ref{fig:dtexample}.},label={lst:p4flow},frame=tb]

...

table classtable {

    key = {
        meta.class: exact;
    }
    actions = {
        forward_by_class;
        ...
    }
    size = 512;
    default_action = NoAction;
}

action hash() {
    meta.hash = 

} 

apply {
    // extract and aggregate features from the packet's headers. The extracted features are hdr.ipv4.totalLen and hdr.tcp.dstPort
    extract_features(); 

    // compute a per-flow identifier by hashing (source IP, source Port, destination IP, destination Port, IPv4 Protocol)
    hash();  
    
    // update the per-flow features according to the packet's features and flow identifier. It also retrieves the flow's updated features
    update_features(); 
   
    // Apply the decision tree model to the flow
    if (meta.NumPackets <= 16.5)
        meta.class = 0;
    else    
        if (meta.MinPacketSize <= 42.0)
            meta.class = 1;
        else    
            if (meta.TcpDstPort <= 109.5)
                meta.class = 1;
            else
                meta.class = 0;
    
    classtable.apply();                
}

...

\end{lstlisting}


\begin{lstlisting}[float=t,caption={{Main portions of the P4 code obtained by applying the \emph{ML to P4 Compiler} component to the per-packet tree of Figure \ref{fig:dtexample}}.},label={lst:p4pkt},frame=tb]

...

table classtable {

    key = {
        meta.class: exact;
    }
    actions = {
        forward_by_class;
        ...
    }
    size = 512;
    default_action = NoAction;
}

...

apply {

    // Apply the decision tree model to the the packet. In this case, features are extracted directly from the packet's headers
    if (hdr.ipv4.totalLen <= 658.5)
        if (hdr.tcp.dstPort <= 137.0)
            if (hdr.ipv4.protocol <= 11.5)
                meta.class = 1;
            else
                meta.class = 0;
        else
            meta.class = 0;
    else
        meta.class = 0;
    
    classtable.apply();                
}

...

\end{lstlisting}

\fi



For instance, suppose that we are interested in a model for classifying packets. Figure \ref{fig:dtexample} shows an example of a decision tree model for three features (\emph{PacketSize}, \emph{TcpDstPort} and \emph{Protocol}) and two classes (\emph{class 0} and \emph{class 1}). When a packet needs to be classified (right of Figure \ref{fig:dtexample}), the first step is to extract the necessary features, i.e., packet size, IPv4 protocol, and TCP destination port, from its headers; the second step is to use theses values to traverse the tree, from root to leaf, respecting the conditions stated in each tree-node; and finally, the last step is to yield the reached leaf's label.

In general, decision tree models are capable of separating non-linear problem, but in some cases, this classifier may not be as powerful as other more complex approaches (e.g., a neural network with several hidden layers). If during the training phase one realizes that a decision tree is not an appropriate choice, then our framework can be easily extended to use a Random Forest model instead. The classification time will grow on the number of trees, but it is also expected to yield more accurate results.

\subsection{ML to P4 Compiler}\label{sub:compiler}

The decision tree is a classic example of a strict binary tree containing $n$ leaves and $2n-1$ nodes. 
Classifying an element $\mathbf{x}$ with a decision tree involves traversing the tree from root to leaf, respecting the conditions in each node until reaching a leaf. When a leaf is reached, its label is returned.
Traditionally, in general-purpose programming languages, this process can be easily implemented by using recursion or repetition loops. However, neither of these options are available in the P4 language. Hence, an alternative is to hard-code the tests and labels within the tree-nodes into \emph{if} and \emph{else} statements.

To that end, the \emph{ML to P4 Compiler} component implements a recursive algorithm responsible for traversing the previously trained decision tree in pre-order fashion. The algorithm adds to the P4 code the statement \texttt{if ($<$condition$>$)} if the node being processed is not a leaf. Then, the left subtree represents the path to be followed when \texttt{$<$condition$>$} evaluates to \emph{true}, while the right subtree adds an \texttt{else} to the code, representing the path to be followed when \texttt{$<$condition$>$} evaluates to \emph{false}. 
A leaf node holds the class to which it belongs, and when this node is reached, a new entry in the code updates the state variable of the current classification.

For instance, the per-flow model presented in Figure \ref{fig:dtexample} (left) is transformed, by the \emph{ML to P4 Compiler} component, into the \emph{if-else} chain contained in Listing~\ref{lst:p4flow}. This Listing also shows how every packet traversing the device is processed. Whenever a new packet arrives, the relevant features from its headers are identified and saved into the pipeline's metadata. 
Then, a hash identifier that maps the packet to its flow is computed. This hash is calculated based on the source and destination IP addresses, source and destination ports, and the IPv4 Protocol field from the packet's header. Before the classification, the flow's features are updated in the TCAM memory registers.
After updating the flow's features, a classification occurs in the following \emph{if-else} chain. Eventually, the flow's class will be stored in the \emph{meta.class} variable.
Next, the table \emph{classtable} drives the packet to the match-action rules defined by the \emph{Controller} component, according to each class and the forwarding or dropping policies.

Similarly to Listing~\ref{lst:p4flow}, Listing~\ref{lst:p4pkt} presents the result of applying the \emph{ML to P4 Compiler} component to the per-packet decision tree of Figure \ref{fig:dtexample}.
When a new package is received in the pipeline, the relevant features are identified.
Differently from the per-flow case, there is no need to compute hash functions or update flow's features. In fact, the packet's features are accessed directly from its headers.
When the \emph{if-else} chain is finalized, the packet's class is stored in the \emph{meta.class} variable.
Finally, the pipeline is then directed to apply the \emph{classtable} match-action rules defined by the \emph{Controller} component according to each class and the forwarding or dropping policies.



    
\section{Case Study: Intrusion Detection} \label{sec:case}

In this section, we present a use case for validation of our framework with a scenario of IDS (\emph{Intrusion Detection System}). More specifically, we show that a simple machine learning model (a single decision tree) can be deployed into a SmartNIC to accurately detect different types of attacks (Section \ref{sub:classres}) with acceptable latency degradation (Section \ref{sub:latency}). We describe the dataset that we used, the challenges involved, and our experimental setup in Sections \ref{sub:dataset}, \ref{sub:challenges}, and \ref{sub:expset}, respectively. We emphasize that all our code will be made publicly available upon the publication of this manuscript.

\subsection{Dataset}\label{sub:dataset}

To conduct our experiments, we relied on the dataset created and made available by Sharafaldin \emph{et al.} \cite{sharafaldin2018toward}. 
The dataset is composed of PCAP files related to network traffic generated in five days (Monday to Friday), whose flows were labeled as being benign or a specific type of attack. The authors argue that they addressed many of the issues related to older, and possibly outdated, datasets.
Moreover, they show that machine learning can be used to accurately distinguish the types of flows by applying well-known algorithms and carefully-designed feature extraction/selection strategies.
It is not our goal to dig into the details about the characteristics of each type of flow, how they were generated, or how they were used for classification purposes, and we refer the interested reader to the aforementioned manuscript for more information.

From the available data, we selected the PCAP files from two days (Wednesday and Thursday), which contain flows related to seven types (or classes), one \emph{benign} (BE) and six attacks: \emph{DOS GoldenEye} (GE), \emph{DOS Hulk} (HK), \emph{DOS Slowhttptest} (SH), \emph{DOS Slowloris} (SL), \emph{Web Brute-force} (BF), and \emph{Port Scan} (PS). 
There are other types of attacks in these two days, but we decided to remove them from our analysis due to extremely low frequency.
A brief summary of the dataset that we are using is presented in Table \ref{tab:datasummary}.

\begin{table}[t]
    \centering
    \caption{Dataset Summary}
    \begin{tabular}{|l|c|c|}
    \hline
       Class  & Number of flows & Number of packets \\
    \hline
      Benign (BE)           & 436,183 & 12,260,490\\
      DOS GoldenEye (GE)    & 7,574 & 66,795 \\
      DOS Hulk (HK)         & 14,108 & 1,245,906\\
      DOS Slowhttptest (SH) & 4,218 & 32,510 \\
      DOS Slowloris (SL)    & 3,894 & 37,236\\
      Web Brute-force (BF)  & 1,356 & 19,755\\
      Port Scan (PS)        & 67,579 & 174,312\\
    \hline
    \end{tabular}
    \label{tab:datasummary}
\end{table}

\subsection{Challenges}\label{sub:challenges}

A natural step to deploy a machine learning model to perform intrusion detection into the data plane would be using the features and models studied in \cite{sharafaldin2018toward}. To that end, one could simply use the P4 language to write code to compute the features and to perform the classification. Unfortunately, there are issues related to software, hardware, and the nature of the application itself that prevent such a direct approach. The most important ones are enumerated below: 
\begin{enumerate}
    \item P4 is not a general-purpose programming language. Therefore, many of the features used in \cite{sharafaldin2018toward} can not be directly (and maybe not efficiently) computed (or adapted) using P4. \label{question:feature}
    \item From a practical point of view, classifying a flow after it ends is not a useful task, because the intrusion may have already happened. An in-network IDS must be able to accurately identify a malicious flow as soon as possible to prevent harm.\label{question:early}
    \item Classifying flows requires keeping per-flow table entries to store the their respective features values. Moreover, these entries must be updated after every package belonging to the flow passes through the hardware where the model is deployed. Therefore, it is necessary to understand whether the overhead of performing a table lookup, entry update, and flow classification is not prohibitive in a high-throughput environment.\label{question:perf}
    \item If instead of classifying flows, due to the cost of the classification, one decides to classify packets individually, then it is not necessary to maintain per-flow table entries. In other words, in a per-packet scenario, the overhead related to table lookup, entry update, and memory usage is not present anymore. However, a single package may be considerably less informative than a flow with regard to the classification task. Hence, it is necessary to understand the trade-off between per-packet and per-flow models.   \label{question:packets}
\end{enumerate}
Next, we present our implementation decisions and experimental methodology to address the four challenges listed above.

\subsection{Experimental setup}\label{sub:expset}

% Please add the following required packages to your document preamble:
% \usepackage{multirow}
\begin{table}[t]
\caption{Per-flow and Per-packet features}
\begin{tabular}{|l|l|p{4.2cm}|}
\hline
\textbf{Model}      & \textbf{Feature} & \textbf{Short description}                                                                                                                                                                \\ \hline
\multirow{11}{*}{Per-flow}  & EtherType        & From Ethernet header                                                                                                                                                                      \\ \cline{2-3} 
                            & Protocol         & From IPv4 header                                                                                                                                                                          \\ \cline{2-3} 
                            & CumIPv4Flag\_$X$ & Cumulative number of occurrences of the IPv4 flag $X$, for each $X\in\{\text{DF}, \text{MF}\}$                                                                                            \\ \cline{2-3} 
                            & TcpDstPort       & TCP destination port                                                                                                                                                                      \\ \cline{2-3} 
                            & UdpDstPort       & UDP destination port                                                                                                                                                                      \\ \cline{2-3} 
                            & CumPacketSize    & Cumulative sum of IPv4 packet size                                                                                                                                                        \\ \cline{2-3} 
                            & FlowDuration     & Time interval since the arrival of the first packet                                                                                                                                       \\ \cline{2-3} 
                            & MaxPacketSize    & Size of the largest packet                                                                                                                                                                \\ \cline{2-3} 
                            & MinPacketSize    & Size of the smallest packet                                                                                                                                                               \\ \cline{2-3} 
                            & NumPackets       & Number of packets                                                                                                                                                                         \\ \cline{2-3} 
                            & CumTcpFlag\_$X$  & Cumulative number of occurrences of the TCP flag $X$, for each $X\in\{\text{FIN}, \text{SYN}, \text{RST}, \text{PSH}, \text{ACK},\text{URG}\}$                                            \\ \hline
\multirow{7}{*}{Per-packet} & EtherType        & From Ethernet header                                                                                                                                                                      \\ \cline{2-3} 
                            & Protocol         & From IPv4 header                                                                                                                                                                          \\ \cline{2-3} 
                            & IPv4Flag\_$X$    & Indicates whether IPv4 flag $X$ is set, for each $X\in\{\text{DF}, \text{MF}\}$                                                                                                           \\ \cline{2-3} 
                            & TcpDstPort       & TCP destination port                                                                                                                                                                      \\ \cline{2-3} 
                            & UdpDstPort       & UDP destination port                                                                                                                                                                      \\ \cline{2-3} 
                            & PacketSize       & IPv4 packet size                                                                                                                                                                          \\ \cline{2-3} 
                            & TcpFlag\_$X$     & Indicates whether the TCP flag $X$ is set, for each $X\in\{\text{FIN}, \text{SYN}, \text{RST}, \text{PSH}, \text{ACK},\text{URG}\}$ 
                                                    \\ \hline
\end{tabular}
\label{tab:feature}
\end{table}





\if 0
% Please add the following required packages to your document preamble:
% \usepackage{multirow}
\begin{table}[ht]
 \centering
\begin{tabular}{lll}
\hline
\multicolumn{1}{|l|}{\textbf{Type of model}}               & \multicolumn{1}{l|}{\textbf{Features}}      & \multicolumn{1}{l|}{\textbf{Description}}                                                                                                                         \\ \hline
\multicolumn{1}{|c|}{\multirow{Per-flow}}  & \multicolumn{1}{l|}{Eher Tpye}     & \multicolumn{1}{l|}{From Ethernet header}                                                                                                                \\ \cline{2-3} 
\multicolumn{1}{|l|}{}                            & \multicolumn{1}{l|}{Protocol}      & \multicolumn{1}{l|}{From IPv4 header}                                                                                                                    \\ \cline{2-3} 
\multicolumn{1}{|l|}{}                            & \multicolumn{1}{l|}{Df}            & \multicolumn{1}{l|}{\begin{tabular}[c]{@{}l@{}}Cumulative number of occurrences \\of the "df" flag in the IPv4 header of \\the flow packets\end{tabular}} \\ \cline{2-3} 
\multicolumn{1}{|l|}{}                            & \multicolumn{1}{l|}{Mf}            & \multicolumn{1}{l|}{\begin{tabular}[c]{@{}l@{}}Cumulative number of occurrences \\of the "mf" flag in the IPv4 header  of \\the flow packets\end{tabular}} \\ \cline{2-3} 
\multicolumn{1}{|l|}{}                            & \multicolumn{1}{l|}{Tcp Dst Port}  & \multicolumn{1}{l|}{Destination port From Tcp header}                                                                                                    \\ \cline{2-3} 
\multicolumn{1}{|l|}{}                            & \multicolumn{1}{l|}{Udp Dst Port}  & \multicolumn{1}{l|}{Destination port From Udp header}                                                                                                    \\ \cline{2-3} 
\multicolumn{1}{|l|}{}                            & \multicolumn{1}{l|}{Total Pkt Len} & \multicolumn{1}{l|}{\begin{tabular}[c]{@{}l@{}}Cumulative packet length in bytes \\from IPv4 header\end{tabular}}                                                                                  \\ \cline{2-3} 
\multicolumn{1}{|l|}{}                            & \multicolumn{1}{l|}{Flow Duration} & \multicolumn{1}{l|}{\begin{tabular}[c]{@{}l@{}}Time interval from the arrival of \\the first packet of the flow to\\the current packet\end{tabular}}     \\ \cline{2-3} 
\multicolumn{1}{|l|}{}                            & \multicolumn{1}{l|}{Max Pkt Len}   & \multicolumn{1}{l|}{Maximum flow packet length}                                                                                                          \\ \cline{2-3} 
\multicolumn{1}{|l|}{}                            & \multicolumn{1}{l|}{Min Pkt Len}   & \multicolumn{1}{l|}{Minimum flow packet length}                                                                                                          \\ \cline{2-3} 
\multicolumn{1}{|l|}{}                            & \multicolumn{1}{l|}{Pkts}          & \multicolumn{1}{l|}{Number of packets in the flow}                                                                                                       \\ \cline{2-3} 
\multicolumn{1}{|l|}{}                            & \multicolumn{1}{l|}{Fin}           & \multicolumn{1}{l|}{\begin{tabular}[c]{@{}l@{}}Cumulative number of occurrences of\\ the "FIN" flag in the Tcp header of \\the flow packets\end{tabular}} \\ \cline{2-3} 
\multicolumn{1}{|l|}{}                            & \multicolumn{1}{l|}{Syn}           & \multicolumn{1}{l|}{\begin{tabular}[c]{@{}l@{}}Cumulative number of occurrences of\\ the "SYN" flag in the Tcp header of \\the flow packets\end{tabular}} \\ \cline{2-3} 
\multicolumn{1}{|l|}{}                            & \multicolumn{1}{l|}{Rst}           & \multicolumn{1}{l|}{\begin{tabular}[c]{@{}l@{}}Cumulative number of occurrences of\\ the "RST" flag in the Tcp header of \\the flow packets\end{tabular}} \\ \cline{2-3} 
\multicolumn{1}{|l|}{}                            & \multicolumn{1}{l|}{Psh}           & \multicolumn{1}{l|}{\begin{tabular}[c]{@{}l@{}}Cumulative number of occurrences of\\ the "PSH" flag in the Tcp header of \\the flow packets\end{tabular}} \\ \cline{2-3} 
\multicolumn{1}{|l|}{}                            & \multicolumn{1}{l|}{Ack}           & \multicolumn{1}{l|}{\begin{tabular}[c]{@{}l@{}}Cumulative number of occurrences of\\ the "ACK" flag in the Tcp header of \\the flow packets\end{tabular}} \\ \cline{2-3} 
\multicolumn{1}{|l|}{}                            & \multicolumn{1}{l|}{Urg}           & \multicolumn{1}{l|}{\begin{tabular}[c]{@{}l@{}}Cumulative number of occurrences of\\ the "URG" flag in the Tcp header of \\the flow packets\end{tabular}} \\ \hline
\multicolumn{1}{|c|}{\multirow{Per-packet}} & \multicolumn{1}{l|}{Eher Tpye}     & \multicolumn{1}{l|}{From Ethernet header}                                                                                                                \\ \cline{2-3} 
\multicolumn{1}{|l|}{}                            & \multicolumn{1}{l|}{Protocol}      & \multicolumn{1}{l|}{From IPv4 header}                                                                                                                    \\ \cline{2-3} 
\multicolumn{1}{|l|}{}                            & \multicolumn{1}{l|}{Tcp Dst Port}  & \multicolumn{1}{l|}{Destination port from Tcp header}                                                                                                    \\ \cline{2-3} 
\multicolumn{1}{|l|}{}                            & \multicolumn{1}{l|}{Udp Dst Port}  & \multicolumn{1}{l|}{Destination port from Udp header}                                                                                                    \\ \cline{2-3} 
\multicolumn{1}{|l|}{}                            & \multicolumn{1}{l|}{Total Len}     & \multicolumn{1}{l|}{From IPv4 header}                                                                                                                    \\ \cline{2-3} 
\multicolumn{1}{|l|}{}                            & \multicolumn{1}{l|}{Flags}         & \multicolumn{1}{l|}{From IPv4 header}                                                                                                                    \\ \cline{2-3} 
\multicolumn{1}{|l|}{}                            & \multicolumn{1}{l|}{Ctrl}          & \multicolumn{1}{l|}{From Tcp header}                                                                                                                     \\ \hline
                                                  &                                    &                                                                                                                                                         
\end{tabular}
\end{table}
\fi

\begin{table*}[t]
\centering{
\caption{Summary of classification results}
\begin{tabular}{llll|lll|lll}
\hline
\hline
                       & \multicolumn{3}{c}{Per-flow - After Last Packet} & \multicolumn{3}{c}{Per-flow - After First Packet} & \multicolumn{3}{c}{Per-packet} \\ \cline{2-10}
                       & Precision & Recall & F1-score & Precision & Recall & F1-score & Precision & Recall & F1-score \\ \cline{2-10}
Benign (BE)            & 0.97 &    0.99 &      0.98   &  0.98 &    0.97 &      0.98      & 1.00           &   0.98     &   0.99       \\
DOS GoldenEye (GE)    & 1.00 &    0.87 &      0.93   &  0.00 &    0.00 &      0.00      & -              &   0.00     &   -          \\
DOS Hulk (HK)         & 1.00 &    1.00 &      1.00   &  -    &    0.00 &      -         & 0.78           &   1.00     &   0.87       \\
DOS Slowhttptest (SH) & 0.87 &    0.94 &      0.90   &  -    &    0.00 &      -         & 0.84           &   0.15     &   0.25       \\
DOS Slowloris (SL)    & 0.99 &    0.95 &      0.97   &  -    &    0.00 &      -         & 0.97           &   0.63     &   0.77       \\
Web Brute-force (BF)  & 0.69 &    0.95 &      0.80   &  0.04 &    1.00 &      0.08      & 0.61           &   0.18     &   0.28       \\
Port Scan (PS)       & 1.00 &    0.94 &      0.97   &  1.00 &    0.94 &      0.97      & 1.00           &   0.71     &   0.83       \\
\textbf{All Attacks combined}   & 0.99 &    0.96 &      0.97   &  0.99 &    0.96 &      0.97      & 0.86           &   0.96     &   0.91        \\
\hline
\hline
\end{tabular}
\label{tab:classres}
}
\end{table*}

\begin{figure*}[t]
    \vspace{0.5cm}
    \centering
    \includegraphics[scale=1.1]{figures/cmlast.pdf}\hspace{0.5cm}
    \includegraphics[scale=1.1]{figures/cmfirst.pdf}\hspace{0.5cm}
    \includegraphics[scale=1.1]{figures/cmpacket.pdf}
    \caption{Confusion matrices for: per-flow classification after the last (left) and the first (middle) packets of each flow are observed; and per-packet classification (right). Values are normalized by the sum of each row.}
    \label{fig:flowclassres}
\end{figure*}

In order to compare the per-packet and per-flow strategies, we decided to create two models, one trained with features from individual packets and another with features from flows. 
Table \ref{tab:feature} shows the features used when building each model. One advantage of working with flows is the possibility of having features which can capture their dynamics (e.g., \emph{duration} and \emph{cumulative packet size} at a given moment), which cannot be accomplished by relying only on packets. One disadvantage of working with flows is the impossibility of directly computing better descriptive measures (e.g., \emph{average}, \emph{variance}, and \emph{standard deviation}) for the time-varying features, since important operations, such as division and square root, are not supported in P4. 
In the per-packet case, our choice of features was motivated by \cite{xiong2019switches}. In fact, we used the same ones, except the IPv6-related features (once the dataset in use only contains IPv4 traffic) and TCP/UDP source port (once they are, in general, randomly chosen by the operating system \cite{larsen2011recommendations}).


In our next step, we extracted two samples from the dataset, one for training and the other for testing the models. To that end, we randomly sampled 100 thousand flows for the training set and another 100 thousand flows for the test set. In both cases, the proportions of flows in each class were set to be the same as in the whole dataset.

Although we used the same training set for the per-flow and per-packet models, the feature extraction process differs significantly in these two cases.
In the per-packet case, we simply extracted the features described in Table \ref{tab:feature} from five, randomly selected, packets from each flow.
The per-flow case is more complicated, once features must be extracted from flows, not packets. For each flow in the training set, we updated the values of the features in Table \ref{tab:feature} after processing each packet of the flow in the same order as they appear in the original PCAP files.
Then, we took five equally-spaced samples from the sequence of features states (we used the percentiles 20\%, 40\%, 60\%, 80\% and 100\% as the observation points to ensure that each flow was sampled at different stages). 

To create the models, we relied on Python's \emph{scikit-learn}\footnote{\url{https://scikit-learn.org/}} implementation of the decision tree classifier, and we used cross-validation to choose the best hyperparameters  (e.g., tree height and minimum number of items per leaf) in order to avoid overfitting.

Finally, we conducted two experiments to test the models. First, we applied the per-packet model to all packets in the test set, and the per-flow model to all flows (each flow was classified once for each one of its packets).
This experiment was performed using a Python script and a BMv2 emulator \cite{p4-bmv2}, and it aimed at assessing the quality of the models (according to several classification metrics) and observing the behavior of the P4 code in a controlled environment. 

Our second experiment was designed to understand the overhead added to standard packet processing in order to execute the operations necessary to the classification task (feature extraction, updating features values, and classifying the flow).
To that end, we deployed each model into a Netronome\footnote{\url{https://www.netronome.com/products/agilio-cx/}} SmartNIC (Agilio  CX2x10GbE), connected via PCI Express 3.0 to an Intel(R) Xeon(R) CPU E5-2620 v3 @ 2.40GHz processor and 16GB of RAM. This SmartNIC has two physical and four virtual interfaces. In this experiment, we randomly selected 100 thousand packets and we sent these packets, one at a time, to the virtual interface \texttt{vf0\_0}. Then, inside the SmartNIC, each packet was handled according to four versions of our P4 application:
\begin{enumerate}
    \item The first version simply forwards the packet to the virtual interface \texttt{vf0\_1}. The purpose of this version is to serve as baseline.
    \item The second version contains the per-packet model. It extracts the per-packet features listed in Table \ref{tab:feature}, classifies the packet, and then forwards it to the virtual interface \texttt{vf0\_1}.
    \item The third version contains the per-flow model. It extracts the per-flow features from the packet, updates the features values related to the flow containing the packet, and forwards the packet to the virtual interface \texttt{vf0\_1}.
    \item Finally, the fourth version is similar to the third one, but it also classifies the flow, with the per-flow model, before forwarding the packet to the virtual interface \texttt{vf0\_1}.
\end{enumerate}
In all these cases, we computed the time different between the instant immediately after the packet enters the virtual interface \texttt{vf0\_0} and the instant right before the packet leaves the SmartNIC via virtual interface \texttt{vf0\_1}.

\subsection{Classification Results}\label{sub:classres}



% talk about most relevant rules...
%\gtodo{Bruno:}{I think it would be interesting here for us to have a example of a tree, or a part of one.}

Table \ref{tab:classres} and Figure \ref{fig:flowclassres} summarize the classification results for our per-flow and per-packet models, including the \emph{confusion matrices}, \emph{precision}, \emph{recall}, and \emph{F1-Score}.
We used the per-flow model in two scenarios: for classifying each flow after it finishes (i.e., after its \emph{last} packet is observed) and to classify each flow after it is just born (i.e., after its \emph{first} packet is observed). The former is not useful in practice, but it gives an upper bound with regard to the quality that our models can achieve. Similarly, the latter gives us a lower bound.
Moreover, the last row of Table \ref{tab:classres} contains the results for the binary version of the problem, where the model is used to distinguish between \emph{benign} and \emph{attack} (regardless of the type of attack).


On one hand, it is possible to observe that classifying a flow after it finishes yields high F1-score values. In fact, these numbers are close to the results obtained in \cite{sharafaldin2018toward}, when a larger number of (more complex) features was available. On the other hand, the same is not true when the classification is performed right after a new flow appears. The \emph{benign} class still has high F1-score, but most of the attacks are incorrectly classified as \emph{Web Brute-force}. These numbers may suggest that classifying a 
``young'' flow is a nearly-infeasible task, but it is important to emphasize that if one is interested only in distinguishing between \emph{benign} and \emph{attack}, then a 0.97 F1-score is achieved.

Motivated by the possibility of classifying ``young'' flows, we computed the number of packets that need to be observed in each flow in order to obtain the correct classification result for the first time (given that such correct result is eventually obtained). As shown in Figure \ref{fig:flowuntil}, with the exception of the class \emph{DOS GoldenEye}, with five packets, most of the flows are properly labeled. Moreover, for some classes, including \emph{benign}, observing one or two packets is enough.

The per-packet model has similar results to the second per-flow scenario (after observing only the first packet). In other words, the per-packet model is not capable of distinguishing different types of attacks, but it yields reasonable results in the \emph{benign} \emph{vs.} \emph{attack} problem, although a considerable fraction of malicious packets are labeled as \emph{benign}.

One issue that can arise with a packet-based model is the \emph{fragmentation} of flows. Suppose that a packet of a legit TCP flow is incorrectly labeled as malicious. This scenario may lead to unexpected (and potentially disastrous) behaviour if, for instance, the network is careless programmed to immediately drop such a packet.
Hence, it is important to understand whether our per-packet model classifies packets from the same flow differently. According to Figure \ref{fig:packetclassres}, fragmentation does arise in many cases. However, it is important to notice that in the \emph{benign} class, around 97\% of the flows did not suffer from the fragmentation issue.

\begin{center}
\begin{figure}[t]
    \centering
    \includegraphics{figures/cdfuntil.pdf}
    \caption{Empirical cumulative distributions for number of observed packets within a flow until correct classification result is obtained with the per-flow model. Only computed for flows that were correctly classified after observing its last packet.}
    \label{fig:flowuntil}
\end{figure}
\begin{figure}[t]
    \centering
    \includegraphics{figures/cdffrag.pdf}
    \caption{Empirical cumulative distribution for the fraction of correctly classified packets within the same flow according to the per-packet model.}
    \label{fig:packetclassres}
\end{figure}
\begin{figure}[t]
    \centering
    \includegraphics{figures/cdftime.pdf}
    \caption{Empirical cumulative distribution for the time that a packet takes to traverse (in and out) the whole pipeline at the SmartNIC in different scenarios.}
    \label{fig:cdftime}
\end{figure}
\end{center}




\subsection{Performance Evaluation Results}\label{sub:latency}

Our last set of results concerns the cost of performing in-network flow and packet classification. Figure \ref{fig:cdftime} presents the distributions of the time that a packet resides inside the SmartNIC according to the four versions of our P4 application enumerated in Section \ref{sub:expset}. The time that a packet needs to be forwarded from \texttt{vf0\_0} to \texttt{vf0\_1} averages around 100ns. The cost added to classify packets, using the per-packet model, is almost negligible. However, a significant overhead arises when we use the per-flow model. The necessary time for extracting the features of a packet and updating its respective flow entry averages around 400ns, and when the classification is performed as well, the total time that takes to process a single packet rises, in average, to about 650ns. 

\begin{figure}
    \vspace{0.3cm}
    \centering
    \includegraphics[scale=0.13,valign=t]{figures/tree-flow.pdf}
    \includegraphics[scale=0.13,valign=t]{figures/tree-pkt.pdf}
    \caption{Structure of each decision tree model: (left) per-flow and  (right) per-packet.}
    \label{fig:tree}
\end{figure}

One interesting aspect of Figure \ref{fig:cdftime} is the difference between the costs of the classification operations performed by the per-packet and per-flow models. Since the per-packet classification overhead is negligible, one should expect that the per-flow classification overhead (without the operations of features extraction and flow update) should be negligible as well. In order to dig deeper into this issue, we looked at the difference between the complexities of these two models. If the per-flow model were significantly more complex than the per-packet one, then a significant cost difference could be explained. However, Figure \ref{fig:tree} shows that, although there is a difference between the two trees (especially with regard to their heights), it is not as significant as the difference in the overheads.
Therefore, there are more intricate reasons (e.g., memory management) responsible for the distinction between the per-flow and per-packet classification costs.

\subsection{Considerations}\label{sub:disc}

The results that we described in this section entail a series of important considerations. Perhaps, the most important one is the clear trade-off between accuracy and efficiency when choosing a per-packet or per-flow model. On one hand, classifying a flow leads to more accurate results, at the cost of keeping a per-flow table. On the other hand, despite having low overhead, the per-packet model has lower accuracy and suffers from the issue of flow fragmentation. In summary, choosing one or the other is a decision that can only be made after carefully analyzing the application requirements. 

In spite of our high-accuracy results, we do not claim that our decision tree models are ready to be deployed in the wild. First, traffic patterns are always changing. Even though the datasets that we used were carefully generated \cite{sharafaldin2018toward}, it is possible that over time they become outdated, and a machine learning model is only as good as the representativeness of the training set. Second, a per-flow model may be memory-intensive since it requires storing flow-related metrics. Eventually, the network device may run out of memory, leading to extreme performance degradation. The solution to both of these problems passes through the interaction between the control and data planes. More specifically, the control plane can assume the role of removing inactive flow-entries from the device's memory and receiving important monitoring information from the data plane. Then, such information can be sent to the knowledge plane in order to help building a better model. 

Another important consideration is related to the limitations of our performance evaluation study. We showed that the latency overhead for the per-packet model is negligible, but the per-flow models inflate the forward time in 6.5x. Despite informative, it is also important to understand the impact of our models on the device's throughput under different scenarios.

Finally, it is possible to alleviate the per-flow classification cost significantly. We can use the fact that few packets are necessary to correctly classify a flow. To do so, we can change our P4 program to stop computing statistics about a flow after observing a predetermined number of packets, deleting its respective table entry, and marking the flow, in another table, accordingly to its inferred class. This new table stores less per-flow bytes and it allows the next packets to be properly labeled without the need of updating the flow statistics and applying the classification operation.

We emphasize that all these limitations and considerations are subjects of our ongoing/future work.




\section{Related Work}\label{sec:rw}

% \gtodo{Giovanni}{begin rw suggestions:}The closes work in the literature to our framework is \cite{xiong2019switches}. However, they do not have per-flow model; we assess the classification results a latency analysis with more details. Also it seems that we have quite different implementation decisions.\gtodo{BRUNO}{I think it is importatnt to be sure here.}
% \gtodo{PS:}{they do have quick quality and performance evaluation at the end of section 6.3}.
% Some suggestions of categories for RW:

Recent years have seen an unprecedented surge in research combining ML and networking. For example, programmable network devices can be used to accelerate neural networks processing \cite{AIAccelerator}, to improve distributed ML through in-network aggregation \cite{DistributedML}, deployed at the edge as virtualized ML functions \cite{MLFV}, congestion control \cite{deeplearning-for-congestioncontrol} and distributed reinforcement learning \cite{Reinf-learning-in-switch}. This paper is complementary to these works, focusing on one specific aspect of ML, classification. 

In terms of traffic classification using ML, there has been a set of recent works \cite{sun2018network, pacheco2018towards, lotfollahi2020deep, dias2019innovative} \cite{xiong2019switches}. 
The closest work in the literature to our framework is \cite{xiong2019switches}. Our approach differs from all previous works, by introducing a per-flow model with fined-grained evaluation. Also, we assess the classification results presenting a latency analysis comparing per-flow and per-packet models, while in \cite{xiong2019switches} they have presented high-level experimental results. To the best of our knowledge, our contribution is a first step toward a pragmatic transformation of decision-tree models into P4 language pipeline.

% with more details. %Also it seems that we have quite different implementation

 %They explored the potential of commodity programmable switches to perform in-networking classification. For this, they mapped a trained model to match-action pipeline into the programmable data plane. However, although they demonstrated in-network classification within a programmable switch and quantified the resource requirements, they do not have a per-flow model with fined-grained evaluation. Also, comparing with them, we assess the classification results in a latency analysis with more details. Moreover, it seems that we have quite different implementation decisions.

%In this ML and networking synergy, many works arise to process ML algorithms bypassing the in-network telemetry (INT) to an external computing resource (e.g., VNF). Using this approach, we would be able to run more refined ML algorithms, by offloading to a more specialized processing unit (e.g., Graphics Processing Unit - GPU). Recently, many works have been going to this direction, such as \cite{Reinf-learning-in-switch, MLFV, DistributedML, deeplearning-for-congestioncontrol, ilievski2020efficiency}. However, differently, our work explores the primitives imposed by the Network Processing Units (NPU) to perform an initial step of forwarding decisions based on a more suitable ML algorithm for this kind of architecture. Further, from one side, this alleviates the network by avoiding unnecessary telemetry information been carried to the network, and to the other, it gives agility in the ML processing (i.e., processing in the scale of nanoseconds).

%\noindent\textbf{Network traffic classification:}this can be ML in networks in general and SDN + ML (there is something written about this below). In both cases, we are different because we are deploying ML to the data plane.

%\noindent\textbf{P4-based filtering:}\gtodo{BRUNO}{Rodolfo shared some papers with us about this.}difference to ours: they do not use any sort of ML.


%\gtodo{Giovanni}{end rw suggestions:}

%\noindent\textbf{In-network machine learning:} \bruno{added}\cite{xiong2019switches, boutaba2018comprehensive}

%Recent years have seen an unprecedented surge in research combining ML and networking. Indeed, it has been shown that network devices can be used to accelerate neural networks processing within programmable network devices \cite{AIAccelerator}, to improve distributed ML through in-network aggregation \cite{DistributedML}, deployed at the edge as virtualized ML functions \cite{MLFV}, congestion control \cite{deeplearning-for-congestioncontrol} and distributed reinforcement learning \cite{Reinf-learning-in-switch}.

%In terms of traffic classification, there has been a set of recent works \cite{sun2018network, pacheco2018towards, lotfollahi2020deep, dias2019innovative} in which the  closest work in the literature to our framework is \cite{xiong2019switches}. However, they do not have per-flow model; we assess the classification results a latency analysis with more details. Also it seems that we have quite different implementation

%This ranges from network   Also, there are works exploring the suitability of programmable network devices to accelerate the AI processing  \cite{In-net-computation, AIAccelerator}. 


%Thus, it is expected that in-network ML would gain much attention from networking community. 


%The implementation of inference within network devices is still considered challenge given that in-network computing consumes resources otherwise required for networking purposes. In this paper, we have introduced a framework for in-network classification focusing in supervised algorithms to a match-action pipeline. 
%Our prototypes are implemented both in software and hardware, and achieve full line rate classifying real world traces. To the best of our knowledge, this is the first step in implementing ML within network devices.



%\\
%\cite{berger2018towards} \bruno{trata aprendizado por reforço para sistemas de cache CDN}
%\\


%The main obstacles include what data can be collected from and what control actions can be exercised on legacy network devices. 
%The ability to program the network by leveraging SDN alleviates these obstacles. The cognition from ML can be used to aid in the automation of network operation and management tasks. Therefore, it is exciting and non-trivial to apply ML techniques for such diverse and complex problems in networking. 

%ML frameworks are being accelerated using network devices [18, 44]

%\cite{sultana2019survey} \bruno{survey about IDS usnig ML and SDN}


%\noindent\textbf{Network traffic classification:}\bruno{added}\cite{sun2018network, pacheco2018towards, lotfollahi2020deep, dias2019innovative}

%\noindent\textbf{P4-based blacklisting:}



\section{Conclusions and Future Work}\label{sec:conc}
In this paper, we have introduced a framework for in-network classification addressing the problem of creating simple and reasonably accurate ML models that can be deployed into the programmable data plane with minor performance degradation. We validate our framework with an IDS use case. We implement decision tree models using the BMv2 emulator \cite{p4-bmv2} and deploy them into a Netronome SmartNIC, as a proof-of-concept. Our results show that high-accuracy is achievable (above  95\%). 

In addition to the specific ongoing/future tasks enumerated in Section \ref{sub:disc}, this work motivates a series of broader questions and research directions. 
First, as in our IDS case study, are there other networking scenarios where flows can be classified appropriately after observing only a few of its packets?
Second, given the optimizations previously discussed, is it possible to use per-flow models in more sophisticated devices (e.g., Barefoot Tofino - TNA)?
Finally, can we combine our in-network classifier with smart agents that decide when to perform new measurements and train/deploy new models?

If these questions can be positively answered, then the work presented in this manuscript can be seen as part of a broader framework. One that is capable of performing complex networking functions, with little human supervision and intervention. % Therefore, our contributions represent a, perhaps small, step towards a future when networks may run themselves. 








% conference papers do not normally have an appendix


% use section* for acknowledgment
%\section*{Acknowledgment}


% trigger a \newpage just before the given reference
% number - used to balance the columns on the last page
% adjust value as needed - may need to be readjusted if
% the document is modified later
%\IEEEtriggeratref{8}
% The "triggered" command can be changed if desired:
%\IEEEtriggercmd{\enlargethispage{-5in}}

% references section

% can use a bibliography generated by BibTeX as a .bbl file
% BibTeX documentation can be easily obtained at:
% http://mirror.ctan.org/biblio/bibtex/contrib/doc/
% The IEEEtran BibTeX style support page is at:
% http://www.michaelshell.org/tex/ieeetran/bibtex/
\bibliographystyle{IEEEtran}
% argument is your BibTeX string definitions and bibliography database(s)
\bibliography{bibliography}
%
% <OR> manually copy in the resultant .bbl file
% set second argument of \begin to the number of references
% (used to reserve space for the reference number labels box)




% that's all folks
\end{document}


