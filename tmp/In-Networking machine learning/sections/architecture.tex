\section{Programmable Data Plane \\ as a Classification Machine }\label{sec:methodology}

The goal of this section is to present the necessary steps to deploy a machine learning model into a programmable networking device by using the P4 language \cite{P4}.
To do so, the ``usual'' machine learning process has to be rethought once we want a fast and accurate classifier, under the limitations of the P4 language expressiveness and of the target hardware resources (e.g., memory and processing power).
Therefore, the success of our proposal depends on approaching the networking traffic classification problem at the right granularity (e.g., packets or flows) and choosing P4-suitable features and models.

Our framework can be seen as a shift from the standard match-action paradigm. Traditionally, switches' actions are driven by table matching, which can range from using only MAC addresses to more complex OpenFlow \cite{OpenFlow} rules. Now, P4 enables us to express the matching as the result of mathematical computations over features that can also be extracted from packet header and computed using e.g. stateful registers. We claim that such computations can be obtained from simple learning algorithms so that they can be helpful in different networking scenarios. One advantage of expressing the matching as a result of a learning algorithm is the possibility of having networking data-driven updates. In other words, if the nature of the traffic changes, new models can be built and deployed into the data plane.

\begin{figure}[t] %!t
\centering
\includegraphics[width=\linewidth]{figures/architecture.pdf}
\caption{Framework's Architecture}
\label{architecture}
\end{figure}

Figure \ref{architecture} shows the high-level architecture of our framework and its main components. The architecture is composed of three planes: 
\begin{enumerate}
    \item The \emph{Knowledge Plane} performs the transformation of knowledge into the a machine learning model. In this context, knowledge may refer to information obtained from external datasets, In-band Network Telemetry, or active measurements. The \emph{Feature Extractor} (Section \ref{sub:features}) component is in charge of selecting and transforming the knowledge base's fields into useful features, considering the limitations of the P4 language.
    The \emph{Machine Learning Model Builder} (Section \ref{sub:builder}) component trains the model based on the features that came from the previous component. 
    
    \item The \emph{Control Plane} connects the Knowledge and Data planes, and it has the goal of mapping the previously trained model to an application, given a target hardware architecture. When the \emph{ML to P4 Compiler} (Section \ref{sub:compiler}) component receives a model from the \emph{Machine Learning Model Builder} component, it creates a new application code based on the P4 language, and/or a new set of match-action rules. Then, the \emph{Firmware builder} component uses the application code to compile the firmware for a specific target architecture and sends it to the \emph{Agent Deployer} component, which, in turn, installs the new firmware into the physical device. Meanwhile, match-action rules, can be sent by the \emph{Controller} to the data plane via a southbound interface.

    \item In the \emph{Data Plane}, when the switch receives the new firmware and the set of match-action rules from the \emph{Control Plane}, it assumes a new behavior previously determined by the machine learning model. At this point, the switch recognizes the class that the packet or flow belongs to and reacts immediately to execute either forwarding or dropping policies.
    
    \if 0
    \item In the \emph{Data Plane}, despite the brand-new firmware generated by the \emph{Firmware Builder}, there is still the need in providing an abstraction layer that does the stitching between the bare-metal switch and control plane. Hence, the \emph{Agent Deployer} component is responsible for installing the new firmware into the device, which allows us to program in a fine-grained way the data plane. Once the firmware is installed, the device's matching behaviour is determined according to the machine learning model.
    \fi
\end{enumerate}

The following sections present details and considerations about the three most important components of our framework:  \emph{Feature Extractor}, \emph{Machine Learning Model Builder}, and \emph{ML to P4 Compiler}.

%\bruno{Smart Match-Action}
%\newline 
%Explain the architecture components